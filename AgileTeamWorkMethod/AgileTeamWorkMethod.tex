\documentclass[11pt,a4paper]{article}
\usepackage[italian]{babel}
\usepackage[utf8]{inputenc}

% in way to comment a block
\long\def\/*#1*/{}

\begin{document}
	\title{Metodo "agile" per lavorare in gruppo .}
	\author{Francesco Rombaldoni}
	\date{}
	\maketitle
	\newpage
	
	\section{Concetti preliminari}
	\subsection{Obiettivo del documento}
	Il documento si pone come \textbf{obiettivo}, quello di fornire \textbf{un metodo per il lavoro di gruppo} (da personalizzare secondo le esigenze) in stile "agile".\\
	
	\subsection{Obiettivo Fondamentale.}
	Si definisce "\textbf{obiettivo fondamentale}" tutto ciò che sia in grado di definire una \textbf{parte fondamentale del problema da risolvere}, come per esempio un capitolo all'interno di un documento da scrivere.\\
	In generale gli "obiettivi fondamentali" si usano per \textbf{tenere traccia del progresso} durante il lavoro.\\
	
	\subsection{Tempo.}
	 Il tempo può essere considerato come una \textbf{risorsa tangibile}, per tanto esso può essere in \textbf{eccesso}, in \textbf{difetto} ed \textbf{accumulabile}.\\
	L'unita temporale normalmente utilizzata è l'ora, essa è composta da 45 minuti di "Sprint" ovvero una fase di sforzo intenso e da 15 minuti di "Cool Down" ovvero un momento di pausa all'interno della quale è consigliato abbandonare la postazione di studio per fare del movimento. \\
	Per quanto riguarda l'amministrazione del tempo vengono consigliati due modelli, il primo è il  modello \textbf{arbitrario}; prevede che dopo 45 minuti di studio intenso ci si fermi arbitrariamente interrompendo il lavoro. Il secondo modello (definito \textbf{libero}) prevede invece di interrompere il lavoro solo dopo il raggiungimento di un certo obiettivo prefissato; in questo caso si consiglia di conteggiare il tempo di "Sprint" fatto e di conseguenza calcolare il tempo di pausa accumulato. (Per esempio dopo due ore di studio consecutive si sono accumulati 30 minuti di pausa).\\
	Il tempo di lavoro giornaliero può essere deciso sulla base \textbf{dell'andamento del progresso}, in particolare si consiglia di tenere una linea massima di 7 ore al giorno e una linea minima di 5 ore al giorno. Si vuole far riflette di come 7 ore per giorno si possono dividere come 3 ore di mattina (se fatte consecutivamente diventa una sessione di studio da 2,25 ore di "Sprint") e 4 ore il pomeriggio (se fatte consecutivamente diventano 3 ore di "Sprint").\\
	
	\subsection{Il lavoro di gruppo come un problema.} 
	Si definisce \textbf{problema} tutto ciò che per essere risolto impone \textbf{l'assunzione di un adattamento particolare}, volto al raggiungimento di un obiettivo. Il lavoro di gruppo ricade pienamente in questa definizione e per tanto può essere visto come un problema.\\
	Il modo migliore per risolvere un problema è suddividerlo in una serie di \textbf{piccole iterazioni} da eseguire fino al raggiungimento \textbf{dell'obiettivo finale}.\\
	Lo scopo è quello di dividere l'intero problema (come il lavoro che dovrà fare in gruppo) in dei \textbf{piccoli problemi}, i quali a loro volta verranno divisi ulteriormente i dei problemi ancora \textbf{più piccoli di nauta iterativa}, in maniera da non solo aumentare il controllo sul problema, ma anche di aumentare la cooperazione all'interno del gruppo di lavoro.\\ 
	
	\subsection{Strumenti}
	Per aumentare il \textbf{coordinamento} tra i singoli componenti del gruppo, si consiglia di utilizzare delle piattaforme che aiutino alla \textbf{pianificazione del lavoro}, un esempio è la piattaforma "slack", la quale permette di riunire in un unico luogo tutte le persone e gli strumenti necessari per svolgere le attività.\\
	Uno dei vantaggi di usare questa piattaforma nello specifico è quello di poter disporre degli \textbf{spazi organizzati}, ovvero la possibilità di aprire un canale di comunicazione (cioè un luogo dove è possibile disporre di tutte le persone, i messaggi e i file relativi all'argomento) per ogni \textbf{obiettivo fondamentale} sul quale ad esempio è possibile lavorare in parallelo in modo da \textbf{aumentare la produttività}. 
	\newpage
	
	\section{Il metodo}
	\subsection{La formazione del gruppo}
	Prima di formare un gruppo di lavoro è bene verificare se le persone che faranno parte del suddetto, posseggano competenze sufficienti per poter adempire alle richieste del lavoro da svolgere.\\
	Questa operazione è molto importante in quanto un problema molto diffuso tra i gruppi di lavoro è che soltanto una piccola parte delle persone che lo compongono posseggono le capacità necessarie per portare a termine il lavoro. Qualora questa condizione si dovesse sviluppare, il carico di lavoro non sarebbe più distribuito equamente tra le componenti del gruppo, ma si otterrà una condizione per la quale una parte del gruppo entrerà in una fase di stress provocata dall'eccessivo carico di lavoro, mentre la restante parte, si troverà in una fase per lo più di nullafacenza.\\
	Ciò comporta inoltre che al momento della consegna, soltanto le persone che sono entrate nella fase di stress saranno in grado di avere una idea precisa di tutto il lavoro svolto, mentre le restanti non saranno in grado di raggiungere questo livello di consapevolezza. Motivo per cui  queste ultime non saranno idonee a presentare il progetto, causando in questo modo un altro effetto maligno.\\
	
	\subsection{Raccolta delle risorse}
	Nella fase della raccolta delle risorse, l'obbiettivo è quello di \textbf{ricercare ed accumulare ordinatamente} tutto il materiale utile alla soluzione del problema, come per esempio nel caso di una ricerca di gruppo, i principali documenti da vagliare per scrivere la suddetta.\\
	Una volta aver formato il gruppo di lavoro, fare una riunione, in modo da poter discutere in maniera più precisa sul lavoro che si vorrebbe fare, facendo molta attenzione a definire e scrivere i vari obbiettivi fondamentali da dover adempire durante il lavoro. Discussi i vari obbiettivi fondamentali, formalizzare i suddetti redigendo una lista accessibile a tutti gli elementi del gruppo, i quali la dovranno inoltre convalidare, in modo da avere la certezza che tutte le persone del progetto siano d'accordo sul lavoro da svolgere.\\
	Avendo ora chiarito la direzione del lavoro di gruppo è importante ricercare il materiale necessario per facilitare il lavoro, oltre che, portare a termine i vari obbiettivi fondamentali.\\
	Uno dei modi migliori per compiere questa azione è quello di avere quanto più possibile materiale in \textbf{forma digitale}, in modo da essere più facilmente condivisibile tramite le piattaforme per il lavoro di gruppo (delle quali si consiglia l'utilizzo).\\
	Il materiale raccolto dovrebbe essere salvato anche all'interno "di un luogo \textbf{sicuro}", ovvero una porzione di memoria all'interno della quale è difficile che i file possano venire cancellati accidentalmente.(Per esempio all'interno di un hard-disk che non sia quello di sistema oppure direttamente in "cloud").\\
	I file raccolti, inoltre, dovrebbero essere salvati e suddivisi in base alla loro \textbf{importanza} oppure \textbf{all'uso}, in modo da poter successivamente risparmiare tempo nella ricerca dei documenti.\\
	Una buona pratica può essere quella di sviluppare (sempre digitalmente) una sorta di mappa nella quale segnare il nome, la locazione e la funzione di ogni documento, in modo da poter disporre di un ulteriore strumento che faciliti la gestione del materiale.\\
	
	\subsection{Pianificazione del lavoro di gruppo} 
	Una volta aver formato il gruppo ed aver disposto i documenti, la fase successiva è la pianificazione del lavoro, l'obbiettivo in questa fase è quello di definire la suddivisione del lavoro e prepara una stima dello sforzo e del tempo richiesto per portare a termine il lavoro di gruppo.\\
	La prima cosa da fare in questa fase è quella di creare una riunione di gruppo nella quale discutere del materiale raccolto, in modo poter prendere coscienza dello sforzo richiesto per portare a termine i vari obbiettivi fondamentali prestabiliti.\\
	Redigere quindi un documento nel quale sia descritta una stima del ritmo di lavoro da tenere per poter portare a termine il progetto senza periodo di stress. questo documento dovrà essere visto e convalidato da ogni elemento del gruppo.
	Una volta fatto ciò, iniziare a discutere riguardo la divisione del lavoro tra i vari membri del gruppo, prestando attenzione allo sforzo personale richiesto per ogni divisione fatta e facendo in modo che ci siano più persone a lavorare sullo stesso argomento, evitando che accada il contrario. Questa cosa non serve soltanto per fare in modo che il carico di lavoro complessivo sia equamente distribuito tra i vari elementi del gruppo, ma inoltre fa sì che ogni argomento venga visionato da più persone, garantendo così una qualità maggiore del materiale prodotto. Conclusa la parte di divisione del lavoro definire inoltre per ogni persona, gli obiettivi settimanali da rispettare per avere un ritmo di lavoro complessivo compatibile con il documento scritto precedentemente.\\
	Tutti i vari obiettivi settimanali dovranno anch'essi essere formalizzati in un documento visibile da tutto il gruppo, permettendo in questo modo che tutti si possano controllare a vicenda durante la fase di lavoro.\\
	
	\subsection{Lavoro di gruppo}
	In questa fase l'obbiettivo è quello di monitorare il lavoro di tutti gli elementi del gruppo, cercando anche di aumentare la collaborazione tra le parti, in modo da aumentare di conseguenza la qualità del prodotto finito.\\
	Una cosa fondamentale in questa fase è quella di stabilire un giorno della settimana nel quale fare una riunione di confronto riguardo l'andamanto del lavoro complessivo, valutando inoltre la necessità di aggiornare le stime precedentemente.\\
	Ogni persona facente parte del gruppo potrà quindi organizzare liberamente il proprio lavoro, oppure, organizzarlo anche confrontandosi con il compagno di lavoro assegnatogli. La cosa più importante è portare termine la scadenza settimanale, in modo da poter essere tutti coordinati, oppure, in modo da poter reagire tempestivamente ad un problema non previsto, siccome in questo modo si riduce la disorganizzazione che potrebbe rallentarne il riconoscimento.\\
	Inoltre durante ogni riunione settimanale è di fondamentale importanza che ogni elemento del gruppo esponga chiaramente il proprio lavoro svolto, in modo da non solo aumentare la coscienza del lavoro complessiva dell'intero gruppo, ma anche, se necessario, far si che tutti gli elementi del gruppo siano in grado di metter mano nel lavoro fatto dalle altre persone.\\
	Questa fase verrà quindi svolta ricorsivamente fino alla soluzione di tutti gli obbiettivi fondamentali precedentemente individuati.\\
	
	\subsection{Conclusione del lavoro}
	Nella fase di conclusione del lavoro, l'obiettivo è quello di ricomporre tutte le parti del lavoro svolte singolarmente da ogni componente del gruppo e di controllare la qualità complessiva dell'intero lavoro svolto.\\
	Una volta aver risolto tutti gli obiettivi fondamentali, organizzare una riunione di gruppo nella quale si dovrà ricomporre tutto il materiale prodotto, per poi stimare lo sforzo necessario per la revisione del suddetto. fatta la stima, bisognerà ridividere il materiale riunito, in modo da poterlo assegnare ai componenti del gruppo per la revisione. (N.B. la nuova suddivisione dovrà fare in modo che le parti non ricadano sulle persone che le hanno precedentemente svolte).\\
	Aspettare quindi una settimana di tempo prima di fare la prossima riunione, in modo da dare modo a tutti i componenti di svolgere correttamente la revisione.\\
	Passata una settimana organizzare nuovamente una riunione nella quale potersi confrontare sulle correzioni apportate, successivamente ricomporre (assieme) nuovamente il materiale e prepararlo per la consegna. 
	
	\section{Ultime considerazioni}
	Come preannunciato questa guida offre delle linee di riferimento per l'organizzazione del lavoro di gruppo. Queste linee guide non sono da considerasi come delle regole da seguire, ma più come uno strumento da personalizzare sulla base delle proprie esperienze e personalità. \\
	
	\end{document}